\documentclass[letterpaper]{book}
\usepackage[times,inconsolata,hyper]{Rd}
\usepackage{makeidx}
\usepackage[utf8]{inputenc} % @SET ENCODING@
% \usepackage{graphicx} % @USE GRAPHICX@
\makeindex{}
\begin{document}
\chapter*{}
\begin{center}
{\textbf{\huge Package `MyCoV'}}
\par\bigskip{\large \today}
\end{center}
\begin{description}
\raggedright{}
\inputencoding{utf8}
\item[Type]\AsIs{Package}
\item[Title]\AsIs{Predict classifications of partial RdRp Coronavirus sequences}
\item[Version]\AsIs{0.1.0}
\item[Author]\AsIs{David A Wilkinson}
\item[Maintainer]\AsIs{David A Wilkinson }\email{dwilkin799@gmail.com}\AsIs{}
\item[Description]\AsIs{This is essentially a wrapper for blastn, which is called to query a sequence database of partial RdRp Coronavirus sequences.
The sequences in the database have been classified by ``predicted subgenus'' based on the level of phylogenetic support of different
clade groups in a bayesian phylogenetic inference and the relative classifications of holotype reference sequences, as detailed in
the associated manuscript.
MyCoV takes any input fasta file (single or multiple sequences), and returns the best hit(s) from the database. It reports the level of
certainty with which the best-matching sequence could be classified by subgenus (again, according to the manuscript), and the pairwise
identity of the sequences you are querying to their reference sequences.
In order to get an idea of whether this means your sequences are likely to belong to the same subgenus, it can also plots the calculated
pairwise distances with all pairwise distances for within- and between- subgenus comparisons for the different genera.
Host and country of origin metadata are also associated with the sequence database. Sequence similarity can be mapped to the phylogeny
from the article, with a visual summary of the similarities and the associated metadata.}
\item[License]\AsIs{MIT + file license}
\item[Encoding]\AsIs{UTF-8}
\item[LazyData]\AsIs{true}
\item[Depends]\AsIs{dplyr,
ape,
ggplot2,
DECIPHER,
seqinr,
formattable,
RColorBrewer,
roxygen2,
magrittr,
grDevices,
phytools,
ggtree}
\item[biocViews]\AsIs{DECIPHER,
Biostrings}
\end{description}
\Rdcontents{\R{} topics documented:}
